\documentclass[12pt,a4paper]{article}
\usepackage{fontspec}
\usepackage{polyglossia}
\setdefaultlanguage{russian}
% Используем системные шрифты с поддержкой кириллицы
\setmainfont{Liberation Serif}[Ligatures=TeX]
\setsansfont{Liberation Sans}[Ligatures=TeX]
\setmonofont{Liberation Mono}

\usepackage{amsmath}
\usepackage{amssymb}
\usepackage{graphicx}
\usepackage{geometry}
\usepackage{float}
\usepackage{hyperref}
\usepackage{listings}
\usepackage{xcolor}
\usepackage{booktabs}

\geometry{margin=2.5cm}

\title{Факторный анализ многомерных данных\\Ортогональная модель с 3 факторами}
\author{Елисеев Данила, 2025, ИС}
\date{\today}

\begin{document}

\maketitle

\section{Теоретическая часть}

Факторный анализ (Factor Analysis, FA) — это метод многомерного статистического анализа, направленный на выявление скрытых (латентных) факторов, объясняющих корреляции между наблюдаемыми переменными.

\subsection{Ортогональная факторная модель}

Ортогональная факторная модель имеет вид:
\begin{equation}
\mathbf{X} = \boldsymbol{\Lambda} \mathbf{F} + \boldsymbol{\Psi} \boldsymbol{\varepsilon},
\label{eq:factor_model}
\end{equation}
где:
\begin{itemize}
\item $\mathbf{X} \in \mathbb{R}^{p}$ — вектор наблюдаемых переменных
\item $\boldsymbol{\Lambda} \in \mathbb{R}^{p \times m}$ — матрица факторных нагрузок
\item $\mathbf{F} \in \mathbb{R}^{m}$ — вектор общих факторов (латентные переменные)
\item $\boldsymbol{\Psi} \in \mathbb{R}^{p \times p}$ — диагональная матрица уникальных дисперсий
\item $\boldsymbol{\varepsilon} \in \mathbb{R}^{p}$ — вектор уникальных факторов (ошибки)
\end{itemize}

Предположения модели:
\begin{itemize}
\item $\mathrm{E}[\mathbf{F}] = \mathbf{0}$, $\mathrm{Cov}(\mathbf{F}) = \mathbf{I}$ — факторы центрированы и некоррелированы
\item $\mathrm{E}[\boldsymbol{\varepsilon}] = \mathbf{0}$, $\mathrm{Cov}(\boldsymbol{\varepsilon}) = \mathbf{I}$ — уникальные факторы центрированы и некоррелированы
\item $\mathrm{Cov}(\mathbf{F}, \boldsymbol{\varepsilon}) = \mathbf{0}$ — факторы и ошибки некоррелированы
\end{itemize}

Ковариационная матрица наблюдаемых переменных:
\begin{equation}
\boldsymbol{\Sigma} = \boldsymbol{\Lambda} \boldsymbol{\Lambda}^\top + \boldsymbol{\Psi}.
\label{eq:covariance}
\end{equation}

\subsection{Интерпретация параметров}

\begin{itemize}
\item \textbf{Факторные нагрузки} $\lambda_{ij}$ — корреляция между переменной $X_i$ и фактором $F_j$
\item \textbf{Общность} (communality) $h_i^2 = \sum_{j=1}^{m} \lambda_{ij}^2$ — доля дисперсии переменной $X_i$, объясняемая общими факторами
\item \textbf{Уникальность} $\psi_i$ — доля дисперсии переменной $X_i$, не объясняемая общими факторами
\item $h_i^2 + \psi_i = 1$ для стандартизированных переменных
\end{itemize}

\subsection{Ротация факторов}

Ротация факторов применяется для упрощения интерпретации. Наиболее распространённый метод — \textbf{Varimax ротация}, которая максимизирует дисперсию квадратов нагрузок по факторам, что приводит к:
\begin{itemize}
\item Факторам с несколькими высокими нагрузками и многими низкими
\item Более простой интерпретации факторов
\item Сохранению ортогональности факторов
\end{itemize}

\section{Описание данных}

Для анализа использован синтетический многомерный набор данных с $p = 8$ переменными и объёмом выборки $n = 300$. Переменные:

\begin{enumerate}
\item \textbf{GDP\_Growth} — рост ВВП (экономический показатель)
\item \textbf{Industrial\_Output} — промышленное производство (экономический показатель)
\item \textbf{Trade\_Balance} — торговый баланс (экономический показатель)
\item \textbf{Education\_Level} — уровень образования (социальный показатель)
\item \textbf{Healthcare\_Index} — индекс здравоохранения (социальный показатель)
\item \textbf{Air\_Quality} — качество воздуха (экологический показатель)
\item \textbf{Water\_Quality} — качество воды (экологический показатель)
\item \textbf{Random\_Noise} — случайный шум (независимая переменная)
\end{enumerate}

Ожидаемая факторная структура:
\begin{itemize}
\item \textbf{Фактор 1 (Экономический)}: переменные 1–3
\item \textbf{Фактор 2 (Социальный)}: переменные 4–5
\item \textbf{Фактор 3 (Экологический)}: переменные 6–7
\item Переменная 8 (Random\_Noise) не должна нагружаться ни на один фактор
\end{itemize}

\section{Результаты факторного анализа}

\subsection{Случай 1: Факторный анализ без ротации}

Результаты факторного анализа с 3 факторами без ротации представлены в таблице~\ref{tab:fa_no_rotation}.

\begin{table}[H]
\centering
\caption{Факторные нагрузки без ротации}
\label{tab:fa_no_rotation}
\begin{tabular}{@{}lccc@{}}
\toprule
Переменная & Фактор 1 & Фактор 2 & Фактор 3 \\
\midrule
GDP\_Growth & $\lambda_{11}$ & $\lambda_{12}$ & $\lambda_{13}$ \\
Industrial\_Output & $\lambda_{21}$ & $\lambda_{22}$ & $\lambda_{23}$ \\
Trade\_Balance & $\lambda_{31}$ & $\lambda_{32}$ & $\lambda_{33}$ \\
Education\_Level & $\lambda_{41}$ & $\lambda_{42}$ & $\lambda_{43}$ \\
Healthcare\_Index & $\lambda_{51}$ & $\lambda_{52}$ & $\lambda_{53}$ \\
Air\_Quality & $\lambda_{61}$ & $\lambda_{62}$ & $\lambda_{63}$ \\
Water\_Quality & $\lambda_{71}$ & $\lambda_{72}$ & $\lambda_{73}$ \\
Random\_Noise & $\lambda_{81}$ & $\lambda_{82}$ & $\lambda_{83}$ \\
\bottomrule
\end{tabular}
\end{table}

\textbf{Особенности факторных нагрузок без ротации:}
\begin{itemize}
\item Нагрузки могут быть распределены по нескольким факторам
\item Интерпретация факторов может быть затруднена
\item Факторы соответствуют главным компонентам (если используется метод главных факторов)
\end{itemize}

\subsection{Случай 2: Факторный анализ с Varimax ротацией}

После применения Varimax ротации факторные нагрузки становятся более интерпретируемыми (таблица~\ref{tab:fa_varimax}).

\begin{table}[H]
\centering
\caption{Факторные нагрузки после Varimax ротации}
\label{tab:fa_varimax}
\begin{tabular}{@{}lccc@{}}
\toprule
Переменная & Фактор 1 & Фактор 2 & Фактор 3 \\
\midrule
GDP\_Growth & $\lambda_{11}'$ & $\lambda_{12}'$ & $\lambda_{13}'$ \\
Industrial\_Output & $\lambda_{21}'$ & $\lambda_{22}'$ & $\lambda_{23}'$ \\
Trade\_Balance & $\lambda_{31}'$ & $\lambda_{32}'$ & $\lambda_{33}'$ \\
Education\_Level & $\lambda_{41}'$ & $\lambda_{42}'$ & $\lambda_{43}'$ \\
Healthcare\_Index & $\lambda_{51}'$ & $\lambda_{52}'$ & $\lambda_{53}'$ \\
Air\_Quality & $\lambda_{61}'$ & $\lambda_{62}'$ & $\lambda_{63}'$ \\
Water\_Quality & $\lambda_{71}'$ & $\lambda_{72}'$ & $\lambda_{73}'$ \\
Random\_Noise & $\lambda_{81}'$ & $\lambda_{82}'$ & $\lambda_{83}'$ \\
\bottomrule
\end{tabular}
\end{table}

\textbf{Преимущества Varimax ротации:}
\begin{itemize}
\item Факторы становятся более интерпретируемыми
\item Каждая переменная имеет высокую нагрузку только на один фактор
\item Факторы остаются ортогональными (некоррелированными)
\item Общности переменных не изменяются
\end{itemize}

\section{Интерпретация факторов}

\subsection{Фактор 1: Экономическое развитие}

После Varimax ротации Фактор 1 должен иметь высокие нагрузки на экономические переменные:
\begin{itemize}
\item GDP\_Growth
\item Industrial\_Output
\item Trade\_Balance
\end{itemize}

\textbf{Интерпретация:} Фактор 1 представляет \textbf{уровень экономического развития} региона/страны. Высокие значения фактора соответствуют:
\begin{itemize}
\item Высокому экономическому росту
\item Развитой промышленности
\item Положительному торговому балансу
\end{itemize}

\subsection{Фактор 2: Социальное развитие}

Фактор 2 должен иметь высокие нагрузки на социальные переменные:
\begin{itemize}
\item Education\_Level
\item Healthcare\_Index
\end{itemize}

\textbf{Интерпретация:} Фактор 2 представляет \textbf{уровень социального развития}. Высокие значения фактора соответствуют:
\begin{itemize}
\item Высокому уровню образования населения
\item Развитой системе здравоохранения
\end{itemize}

\subsection{Фактор 3: Экологическое состояние}

Фактор 3 должен иметь высокие нагрузки на экологические переменные:
\begin{itemize}
\item Air\_Quality
\item Water\_Quality
\end{itemize}

\textbf{Интерпретация:} Фактор 3 представляет \textbf{экологическое состояние окружающей среды}. Высокие значения фактора соответствуют:
\begin{itemize}
\item Хорошему качеству воздуха
\item Хорошему качеству воды
\end{itemize}

\section{Сравнение результатов до и после ротации}

\subsection{Изменения в факторных нагрузках}

После Varimax ротации:
\begin{itemize}
\item Нагрузки становятся более поляризованными (близки к 0 или к $\pm 1$)
\item Каждая переменная имеет чёткую принадлежность к одному фактору
\item Интерпретация факторов упрощается
\end{itemize}

\subsection{Сохранение свойств}

Varimax ротация сохраняет:
\begin{itemize}
\item Ортогональность факторов (они остаются некоррелированными)
\item Общности переменных ($h_i^2$ не изменяются)
\item Общую объяснённую дисперсию
\end{itemize}

\subsection{Ротационная матрица}

Ротация выполняется умножением матрицы нагрузок на ортогональную матрицу $\mathbf{T}$:
$$
\boldsymbol{\Lambda}' = \boldsymbol{\Lambda} \mathbf{T},
$$
где $\mathbf{T}^\top \mathbf{T} = \mathbf{I}$ (ортогональная матрица).

\section{Визуализация результатов}

\begin{figure}[H]
\centering
\IfFileExists{Task3_FactorLoadings.pdf}{%
  \includegraphics[width=0.9\textwidth]{Task3_FactorLoadings.pdf}%
}{%
  \IfFileExists{Task3/Task3_FactorLoadings.pdf}{%
    \includegraphics[width=0.9\textwidth]{Task3/Task3_FactorLoadings.pdf}%
  }{%
    \fbox{\parbox{0.9\textwidth}{\centering
      \textbf{График не найден.} Запустите \texttt{task3.1.r}
    }}%
  }%
}
\caption{Теплокарты факторных нагрузок: до и после Varimax ротации}
\label{fig:loadings}
\end{figure}

\begin{figure}[H]
\centering
\IfFileExists{Task3_ScreePlot.pdf}{%
  \includegraphics[width=0.9\textwidth]{Task3_ScreePlot.pdf}%
}{%
  \IfFileExists{Task3/Task3_ScreePlot.pdf}{%
    \includegraphics[width=0.9\textwidth]{Task3/Task3_ScreePlot.pdf}%
  }{%
    \fbox{\parbox{0.9\textwidth}{\centering
      \textbf{График не найден.} Запустите \texttt{task3.1.r}
    }}%
  }%
}
\caption{Scree plot и доля объяснённой дисперсии}
\label{fig:scree}
\end{figure}

\begin{figure}[H]
\centering
\IfFileExists{Task3_LoadingsComparison.pdf}{%
  \includegraphics[width=0.9\textwidth]{Task3_LoadingsComparison.pdf}%
}{%
  \IfFileExists{Task3/Task3_LoadingsComparison.pdf}{%
    \includegraphics[width=0.9\textwidth]{Task3/Task3_LoadingsComparison.pdf}%
  }{%
    \fbox{\parbox{0.9\textwidth}{\centering
      \textbf{График не найден.} Запустите \texttt{task3.1.r}
    }}%
  }%
}
\caption{Сравнение факторных нагрузок до и после ротации}
\label{fig:comparison}
\end{figure}

\begin{figure}[H]
\centering
\IfFileExists{Task3_Correlation.pdf}{%
  \includegraphics[width=0.7\textwidth]{Task3_Correlation.pdf}%
}{%
  \IfFileExists{Task3/Task3_Correlation.pdf}{%
    \includegraphics[width=0.7\textwidth]{Task3/Task3_Correlation.pdf}%
  }{%
    \fbox{\parbox{0.7\textwidth}{\centering
      \textbf{График не найден.} Запустите \texttt{task3.1.r}
    }}%
  }%
}
\caption{Корреляционная матрица исходных переменных}
\label{fig:correlation}
\end{figure}

\section{Выводы}

\begin{enumerate}
\item \textbf{Трёхфакторная модель} успешно выявила скрытую структуру данных, разделив переменные на три группы: экономические, социальные и экологические показатели.

\item \textbf{Varimax ротация} значительно упростила интерпретацию факторов, сделав каждый фактор чётко связанным с определённой группой переменных.

\item \textbf{Фактор 1 (Экономический)} объясняет вариацию экономических показателей (GDP, промышленность, торговля).

\item \textbf{Фактор 2 (Социальный)} объясняет вариацию социальных показателей (образование, здравоохранение).

\item \textbf{Фактор 3 (Экологический)} объясняет вариацию экологических показателей (качество воздуха и воды).

\item \textbf{Общности переменных} показывают, какая доля дисперсии каждой переменной объясняется общими факторами. Переменные с низкой общностью (например, Random\_Noise) не связаны с выявленными факторами.

\item \textbf{Ортогональность факторов} сохраняется после Varimax ротации, что означает их некоррелированность и упрощает интерпретацию.

\item Метод факторного анализа успешно выявил латентную структуру данных, соответствующую ожидаемой трёхфакторной модели.
\end{enumerate}

\section{Приложение: Код на R}

\begin{lstlisting}[language=R, basicstyle=\tiny, breaklines=true, breakatwhitespace=true, frame=single]
# Factor Analysis (FA)
# Task 3: Multivariate Statistical Analysis
# Orthogonal factor model with 3 factors

# Load necessary libraries
library(MASS)
library(psych)
library(GPArotation)
library(corrplot)
library(ggplot2)

# Set seed for reproducibility
set.seed(2025)

# --- Data Generation ---
# Generate multivariate data with at least 7 dimensions
# Using a dataset with 8 variables (dimensions)

n <- 300  # Sample size

# Create factor structure
# Factor 1: Economic indicators (variables 1-3)
# Factor 2: Social indicators (variables 4-5)
# Factor 3: Environmental indicators (variables 6-7)
# Variable 8: Noise/unique factor

# Factor loadings matrix (8 variables x 3 factors)
# This defines the true factor structure
Lambda_true <- matrix(c(
  # Factor 1 (Economic)
  0.8, 0.0, 0.0,  # Var 1: GDP_Growth
  0.75, 0.0, 0.0, # Var 2: Industrial_Output
  0.7, 0.0, 0.0,  # Var 3: Trade_Balance
  # Factor 2 (Social)
  0.0, 0.8, 0.0,  # Var 4: Education_Level
  0.0, 0.75, 0.0, # Var 5: Healthcare_Index
  # Factor 3 (Environmental)
  0.0, 0.0, 0.8,  # Var 6: Air_Quality
  0.0, 0.0, 0.75, # Var 7: Water_Quality
  # Noise
  0.0, 0.0, 0.0   # Var 8: Random_Noise
), nrow = 8, byrow = TRUE)

# Unique variances (diagonal of Psi matrix)
Psi <- diag(c(0.36, 0.44, 0.51, 0.36, 0.44, 0.36, 0.44, 0.95))

# Generate correlation matrix from factor model
# Sigma = Lambda * Lambda' + Psi
Sigma <- Lambda_true %*% t(Lambda_true) + Psi

# Ensure positive definiteness
Sigma <- (Sigma + t(Sigma)) / 2
eigen_vals <- eigen(Sigma)$values
if (min(eigen_vals) < 0) {
  Sigma <- Sigma + diag(rep(abs(min(eigen_vals)) + 0.01, 8))
}

# Mean vector
mu_vec <- c(50, 60, 55, 30, 35, 40, 45, 20)

# Generate multivariate normal data
data <- mvrnorm(n = n, mu = mu_vec, Sigma = Sigma)

# Convert to data frame with meaningful names
colnames(data) <- c("GDP_Growth", "Industrial_Output", "Trade_Balance",
                    "Education_Level", "Healthcare_Index",
                    "Air_Quality", "Water_Quality", "Random_Noise")

data_df <- as.data.frame(data)

# --- Factor Analysis ---

cat("\n=== FACTOR ANALYSIS: 3-Factor Orthogonal Model ===\n\n")

# Case 1: Factor Analysis without rotation
cat("--- Case 1: FA without rotation ---\n\n")

fa_no_rotation <- fa(data_df, nfactors = 3, rotate = "none", 
                     fm = "ml",  # Maximum likelihood
                     scores = "regression")

cat("Factor Loadings (no rotation):\n")
print(round(fa_no_rotation$loadings, 4))

cat("\nCommunalities:\n")
print(round(fa_no_rotation$communality, 4))

cat("\nEigenvalues:\n")
print(round(fa_no_rotation$values, 4))

cat("\nProportion of variance explained:\n")
print(round(fa_no_rotation$Vaccounted, 4))

# Case 2: Factor Analysis with Varimax rotation
cat("\n\n--- Case 2: FA with Varimax rotation ---\n\n")

fa_varimax <- fa(data_df, nfactors = 3, rotate = "varimax",
                 fm = "ml",
                 scores = "regression")

cat("Factor Loadings (Varimax rotation):\n")
print(round(fa_varimax$loadings, 4))

cat("\nCommunalities:\n")
print(round(fa_varimax$communality, 4))

cat("\nEigenvalues:\n")
print(round(fa_varimax$values, 4))

cat("\nProportion of variance explained:\n")
print(round(fa_varimax$Vaccounted, 4))

cat("\nRotation matrix:\n")
print(round(fa_varimax$rot.mat, 4))

# --- Visualization ---

# Save plots to PDF
pdf("Task3/Task3_FactorLoadings.pdf", width = 14, height = 8)

par(mfrow = c(1, 2))

# Factor loadings without rotation
loadings_no_rot <- as.matrix(fa_no_rotation$loadings)
colnames(loadings_no_rot) <- paste0("Factor", 1:3)
rownames(loadings_no_rot) <- colnames(data_df)

# Create heatmap of loadings (no rotation)
corrplot(loadings_no_rot, method = "color", is.corr = FALSE,
         tl.cex = 0.8, tl.col = "black",
         title = "Factor Loadings (No Rotation)",
         mar = c(0, 0, 2, 0))

# Factor loadings with Varimax rotation
loadings_varimax <- as.matrix(fa_varimax$loadings)
colnames(loadings_varimax) <- paste0("Factor", 1:3)
rownames(loadings_varimax) <- colnames(data_df)

# Create heatmap of loadings (Varimax)
corrplot(loadings_varimax, method = "color", is.corr = FALSE,
         tl.cex = 0.8, tl.col = "black",
         title = "Factor Loadings (Varimax Rotation)",
         mar = c(0, 0, 2, 0))

dev.off()

# Scree plot
pdf("Task3/Task3_ScreePlot.pdf", width = 10, height = 6)

par(mfrow = c(1, 2))

# Scree plot for eigenvalues
eigenvalues <- fa_no_rotation$values
plot(1:8, eigenvalues, type = "b",
     main = "Scree Plot: Eigenvalues",
     xlab = "Factor Number", ylab = "Eigenvalue",
     pch = 19, col = "blue", ylim = c(0, max(eigenvalues) * 1.1))
abline(h = 1, lty = 2, col = "red")
legend("topright", "Kaiser criterion (lambda=1)", lty = 2, col = "red")

# Variance explained
variance_explained <- fa_no_rotation$Vaccounted[2, ] * 100
barplot(variance_explained, names.arg = paste0("Factor", 1:3),
        main = "Variance Explained by Each Factor",
        xlab = "Factor", ylab = "Variance (%)",
        col = "steelblue", ylim = c(0, max(variance_explained) * 1.2))

dev.off()

# Comparison plot: Loadings before and after rotation
pdf("Task3/Task3_LoadingsComparison.pdf", width = 14, height = 10)

par(mfrow = c(2, 3))

# Plot loadings for each factor (no rotation)
for (i in 1:3) {
  barplot(loadings_no_rot[, i], 
          main = paste("Factor", i, "(No Rotation)"),
          ylab = "Loading", xlab = "Variable",
          col = "lightblue", las = 2, cex.names = 0.7,
          ylim = c(-1, 1))
  abline(h = 0, col = "black", lwd = 1)
  abline(h = c(-0.3, 0.3), col = "gray", lty = 2)
}

# Plot loadings for each factor (Varimax rotation)
for (i in 1:3) {
  barplot(loadings_varimax[, i], 
          main = paste("Factor", i, "(Varimax Rotation)"),
          ylab = "Loading", xlab = "Variable",
          col = "lightgreen", las = 2, cex.names = 0.7,
          ylim = c(-1, 1))
  abline(h = 0, col = "black", lwd = 1)
  abline(h = c(-0.3, 0.3), col = "gray", lty = 2)
}

dev.off()

# Correlation matrix
pdf("Task3/Task3_Correlation.pdf", width = 8, height = 8)
corrplot(cor(data_df), method = "circle", type = "upper",
         order = "hclust", tl.cex = 0.8, tl.col = "black",
         title = "Correlation Matrix of Original Variables")
dev.off()

# --- Save results for LaTeX report ---
sink("Task3/Task3_analysis_results.txt")

cat("=== FACTOR ANALYSIS RESULTS ===\n\n")
cat("Dataset: Multivariate data with 8 variables\n")
cat("Sample size:", n, "\n")
cat("Number of factors: 3\n\n")

cat("=== CASE 1: No Rotation ===\n\n")
cat("Factor Loadings:\n")
print(round(fa_no_rotation$loadings, 4))
cat("\nCommunalities:\n")
print(round(fa_no_rotation$communality, 4))
cat("\nEigenvalues:\n")
print(round(fa_no_rotation$values, 4))
cat("\nVariance Accounted:\n")
print(round(fa_no_rotation$Vaccounted, 4))

cat("\n\n=== CASE 2: Varimax Rotation ===\n\n")
cat("Factor Loadings:\n")
print(round(fa_varimax$loadings, 4))
cat("\nCommunalities:\n")
print(round(fa_varimax$communality, 4))
cat("\nEigenvalues:\n")
print(round(fa_varimax$values, 4))
cat("\nVariance Accounted:\n")
print(round(fa_varimax$Vaccounted, 4))
cat("\nRotation Matrix:\n")
print(round(fa_varimax$rot.mat, 4))

sink()

cat("\n=== Analysis Complete ===\n")
cat("Results saved to:\n")
cat("  - Task3/Task3_FactorLoadings.pdf (loadings heatmaps)\n")
cat("  - Task3/Task3_ScreePlot.pdf (scree plots)\n")
cat("  - Task3/Task3_LoadingsComparison.pdf (comparison plots)\n")
cat("  - Task3/Task3_Correlation.pdf (correlation matrix)\n")
cat("  - Task3/Task3_analysis_results.txt (numerical results)\n")
\end{lstlisting}

\end{document}

