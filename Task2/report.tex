\documentclass[12pt,a4paper]{article}
\usepackage{fontspec}
\usepackage{polyglossia}
\setdefaultlanguage{russian}
% Используем системные шрифты с поддержкой кириллицы
\setmainfont{Liberation Serif}[Ligatures=TeX]
\setsansfont{Liberation Sans}[Ligatures=TeX]
\setmonofont{Liberation Mono}

\usepackage{amsmath}
\usepackage{amssymb}
\usepackage{graphicx}
\usepackage{geometry}
\usepackage{float}
\usepackage{hyperref}
\usepackage{listings}
\usepackage{xcolor}
\usepackage{booktabs}

\geometry{margin=2.5cm}

\title{Анализ главных компонент (PCA)\\многомерных данных}
\author{Елисеев Данила, 2025, ИС}
\date{\today}

\begin{document}

\maketitle

\section{Теоретическая часть}

Метод главных компонент (Principal Component Analysis, PCA) — это метод снижения размерности данных, который находит линейные комбинации исходных переменных, максимизирующие дисперсию.

Для центрированных и стандартизированных данных $\mathbf{X} \in \mathbb{R}^{n \times p}$ главная компонента $Y_k$ имеет вид:
$$
Y_k = \mathbf{e}_k^\top \mathbf{X} = \sum_{j=1}^{p} e_{jk} X_j,
$$
где $\mathbf{e}_k$ — $k$-й собственный вектор ковариационной матрицы $\boldsymbol{\Sigma}$ (или корреляционной матрицы $\mathbf{R}$), соответствующий $k$-му по величине собственному значению $\lambda_k$.

Свойства главных компонент:
\begin{itemize}
\item $\mathrm{Var}(Y_k) = \lambda_k$ — дисперсия $k$-й ГК равна соответствующему собственному значению
\item $\mathrm{Cov}(Y_i, Y_j) = 0$ при $i \neq j$ — главные компоненты некоррелированы
\item $\sum_{k=1}^{p} \lambda_k = \sum_{j=1}^{p} \mathrm{Var}(X_j) = p$ (для стандартизированных данных)
\end{itemize}

Доля объяснённой дисперсии $k$-й главной компонентой:
$$
\frac{\lambda_k}{\sum_{j=1}^{p} \lambda_j} = \frac{\lambda_k}{p}.
$$

\section{Описание данных}

Для анализа использован синтетический многомерный набор данных с $p = 6$ переменными и объёмом выборки $n = 200$. Переменные:

\begin{enumerate}
\item \textbf{GDP\_Growth} — рост ВВП (экономический показатель)
\item \textbf{Industrial\_Output} — промышленное производство (экономический показатель)
\item \textbf{Trade\_Balance} — торговый баланс (экономический показатель)
\item \textbf{Education\_Level} — уровень образования (социальный показатель)
\item \textbf{Healthcare\_Index} — индекс здравоохранения (социальный показатель)
\item \textbf{Random\_Factor} — случайный фактор (независимая переменная)
\end{enumerate}

Структура корреляций:
\begin{itemize}
\item Переменные 1–3 (экономические) сильно коррелированы между собой ($\rho \approx 0.6$–$0.7$)
\item Переменные 4–5 (социальные) сильно коррелированы ($\rho \approx 0.75$)
\item Переменная 6 (случайный фактор) независима от остальных
\end{itemize}

\section{Результаты анализа}

\subsection{Случай 1: Анализ с 2 главными компонентами}

Результаты PCA с двумя главными компонентами представлены в таблице~\ref{tab:pca2}.

\begin{table}[H]
\centering
\caption{Результаты PCA (все компоненты)}
\label{tab:pca2}
\begin{tabular}{@{}lcccc@{}}
\toprule
Компонента & Собственное значение & Доля дисперсии (\%) & Накопленная доля (\%) \\
\midrule
PC1 & 2.2513 & 37.52 & 37.52 \\
PC2 & 1.7173 & 28.62 & 66.14 \\
PC3 & 0.9963 & 16.61 & 82.75 \\
PC4 & 0.4887 & 8.14 & 90.89 \\
PC5 & 0.3181 & 5.30 & 96.20 \\
PC6 & 0.2282 & 3.80 & 100.00 \\
\bottomrule
\end{tabular}
\end{table}

\subsubsection{Интерпретация первой главной компоненты (PC1)}

Первая главная компонента объясняет наибольшую долю дисперсии — $37.52\%$ (собственное значение $\lambda_1 = 2.2513$). 

\textbf{Структура нагрузок PC1:}
\begin{itemize}
\item Высокие положительные нагрузки на экономические переменные:
  \begin{itemize}
  \item GDP\_Growth: 0.5541
  \item Industrial\_Output: 0.5961
  \item Trade\_Balance: 0.5594
  \end{itemize}
\item Низкие нагрузки на социальные переменные (Education\_Level: 0.0823, Healthcare\_Index: 0.1130)
\item Очень низкая нагрузка на Random\_Factor: -0.0725
\end{itemize}

\textbf{Интерпретация:} PC1 представляет общий \textbf{уровень экономического и социального развития}. Высокие значения PC1 соответствуют странам/регионам с:
\begin{itemize}
\item Высоким экономическим ростом
\item Развитой промышленностью
\item Положительным торговым балансом
\item Высоким уровнем образования и здравоохранения
\end{itemize}

PC1 можно назвать \textbf{«Индекс общего развития»} или \textbf{«Комплексный показатель благосостояния»}.

\subsubsection{Интерпретация второй главной компоненты (PC2)}

Вторая главная компонента объясняет $28.62\%$ дисперсии (собственное значение $\lambda_2 = 1.7173$) и ортогональна PC1.

\textbf{Структура нагрузок PC2:}
\begin{itemize}
\item Высокие положительные нагрузки на социальные переменные:
  \begin{itemize}
  \item Education\_Level: 0.6970
  \item Healthcare\_Index: 0.6975
  \end{itemize}
\item Низкие (отрицательные) нагрузки на экономические переменные:
  \begin{itemize}
  \item GDP\_Growth: -0.1461
  \item Industrial\_Output: -0.0723
  \item Trade\_Balance: -0.0245
  \end{itemize}
\item Очень низкая нагрузка на Random\_Factor: -0.0208
\end{itemize}

\textbf{Интерпретация:} PC2 отражает \textbf{дисбаланс между экономическим и социальным развитием}. Высокие значения PC2 могут означать:
\begin{itemize}
\item Преобладание экономических показателей над социальными (или наоборот)
\item Различные модели развития (экономически-ориентированные vs социально-ориентированные)
\end{itemize}

PC2 можно назвать \textbf{«Индекс структурного баланса»} или \textbf{«Показатель приоритетов развития»}.

\subsection{Случай 2: Анализ с 3 главными компонентами}

При включении третьей главной компоненты накопленная доля объяснённой дисперсии увеличивается до $75$–$85\%$.

\subsubsection{Интерпретация третьей главной компоненты (PC3)}

Третья главная компонента объясняет $16.61\%$ дисперсии (собственное значение $\lambda_3 = 0.9963$).

\textbf{Структура нагрузок PC3:}
\begin{itemize}
\item Очень высокая нагрузка на Random\_Factor: 0.9946
\item Низкие нагрузки на все остальные переменные (все $< 0.07$)
\item Экономические переменные: GDP\_Growth (0.0619), Industrial\_Output (0.0673), Trade\_Balance (-0.0139)
\item Социальные переменные: Education\_Level (0.0018), Healthcare\_Index (0.0472)
\end{itemize}

\textbf{Интерпретация:} PC3 практически полностью определяется \textbf{случайным фактором} (Random\_Factor), который не коррелирует с остальными переменными. Это подтверждает, что Random\_Factor действительно независим от экономических и социальных показателей. PC3 объясняет 16.61\% дисперсии, что соответствует доле дисперсии Random\_Factor в общем наборе данных.

\section{Визуализация результатов}

\begin{figure}[H]
\centering
\IfFileExists{Task2_Results.pdf}{%
  \includegraphics[width=0.9\textwidth]{Task2_Results.pdf}%
}{%
  \IfFileExists{Task2/Task2_Results.pdf}{%
    \includegraphics[width=0.9\textwidth]{Task2/Task2_Results.pdf}%
  }{%
    \fbox{\parbox{0.9\textwidth}{\centering
      \textbf{График не найден.} Запустите \texttt{task2.1.r}
    }}%
  }%
}
\caption{Графики собственных значений и долей объяснённой дисперсии}
\label{fig:scree}
\end{figure}

\begin{figure}[H]
\centering
\IfFileExists{Task2_Biplot.pdf}{%
  \includegraphics[width=0.9\textwidth]{Task2_Biplot.pdf}%
}{%
  \IfFileExists{Task2/Task2_Biplot.pdf}{%
    \includegraphics[width=0.9\textwidth]{Task2/Task2_Biplot.pdf}%
  }{%
    \fbox{\parbox{0.9\textwidth}{\centering
      \textbf{График не найден.} Запустите \texttt{task2.1.r}
    }}%
  }%
}
\caption{Биплоты: проекция данных и переменных на плоскости главных компонент}
\label{fig:biplot}
\end{figure}

\begin{figure}[H]
\centering
\IfFileExists{Task2_Correlation.pdf}{%
  \includegraphics[width=0.7\textwidth]{Task2_Correlation.pdf}%
}{%
  \IfFileExists{Task2/Task2_Correlation.pdf}{%
    \includegraphics[width=0.7\textwidth]{Task2/Task2_Correlation.pdf}%
  }{%
    \fbox{\parbox{0.7\textwidth}{\centering
      \textbf{График не найден.} Запустите \texttt{task2.1.r}
    }}%
  }%
}
\caption{Корреляционная матрица исходных переменных}
\label{fig:correlation}
\end{figure}

\begin{figure}[H]
\centering
\IfFileExists{Task2_Loadings.pdf}{%
  \includegraphics[width=0.9\textwidth]{Task2_Loadings.pdf}%
}{%
  \IfFileExists{Task2/Task2_Loadings.pdf}{%
    \includegraphics[width=0.9\textwidth]{Task2/Task2_Loadings.pdf}%
  }{%
    \fbox{\parbox{0.9\textwidth}{\centering
      \textbf{График не найден.} Запустите \texttt{task2.1.r}
    }}%
  }%
}
\caption{Нагрузки главных компонент на исходные переменные}
\label{fig:loadings}
\end{figure}

\section{Выводы}

\begin{enumerate}
\item \textbf{Первая главная компонента (PC1)} объясняет наибольшую долю дисперсии и представляет общий уровень экономического и социального развития. Это основной фактор, характеризующий общее благосостояние.

\item \textbf{Вторая главная компонента (PC2)} отражает структурные различия между экономическим и социальным развитием, показывая различные модели развития регионов/стран.

\item \textbf{Две главные компоненты} объясняют $66.14\%$ общей дисперсии, что позволяет существенно снизить размерность данных с сохранением основной информации.

\item \textbf{Третья главная компонента} добавляет ещё $16.61\%$ объяснённой дисперсии и практически полностью определяется случайным фактором (Random\_Factor), что подтверждает его независимость от остальных переменных.

\item Метод PCA успешно выявил скрытую структуру данных, разделив переменные на экономические и социальные группы, что соответствует исходной структуре корреляций.
\end{enumerate}

\section{Приложение: Код на R}

\begin{lstlisting}[language=R, basicstyle=\tiny, breaklines=true, breakatwhitespace=true, frame=single]
# Principal Components Analysis (PCA)
# Task 2: Multivariate Statistical Analysis

# Load necessary libraries
library(MASS)
library(ggplot2)
library(corrplot)
library(factoextra)
library(FactoMineR)

# Set seed for reproducibility
set.seed(2025)

# --- Data Generation ---
# Generate multivariate data with at least 5 dimensions
# Using a dataset with 6 variables (dimensions)

n <- 200  # Sample size

# Create correlation structure
# Variables 1-3: correlated group (economic indicators)
# Variables 4-5: correlated group (social indicators)
# Variable 6: independent (random factor)

# Correlation matrix
cor_matrix <- matrix(c(
  1.0, 0.7, 0.6, 0.2, 0.1, 0.0,  # Var 1
  0.7, 1.0, 0.65, 0.15, 0.2, 0.0,  # Var 2
  0.6, 0.65, 1.0, 0.1, 0.15, 0.0,  # Var 3
  0.2, 0.15, 0.1, 1.0, 0.75, 0.0,  # Var 4
  0.1, 0.2, 0.15, 0.75, 1.0, 0.0,  # Var 5
  0.0, 0.0, 0.0, 0.0, 0.0, 1.0     # Var 6
), nrow = 6, byrow = TRUE)

# Standard deviations
sd_vec <- c(2.5, 3.0, 2.8, 1.8, 2.0, 1.5)

# Convert correlation to covariance matrix
cov_matrix <- diag(sd_vec) %*% cor_matrix %*% diag(sd_vec)

# Mean vector
mu_vec <- c(50, 60, 55, 30, 35, 20)

# Generate multivariate normal data
data <- mvrnorm(n = n, mu = mu_vec, Sigma = cov_matrix)

# Convert to data frame with meaningful names
colnames(data) <- c("GDP_Growth", "Industrial_Output", "Trade_Balance", 
                    "Education_Level", "Healthcare_Index", "Random_Factor")

data_df <- as.data.frame(data)

# --- Principal Components Analysis ---

# Case 1: PCA with 2 principal components
cat("\n=== CASE 1: PCA with 2 Principal Components ===\n\n")

pca_2 <- prcomp(data_df, center = TRUE, scale. = TRUE)

# Summary
cat("Summary of PCA (2 components):\n")
print(summary(pca_2))

# Eigenvalues and variance explained
eigenvalues_2 <- pca_2$sdev^2
prop_var_2 <- eigenvalues_2 / sum(eigenvalues_2)
cum_prop_var_2 <- cumsum(prop_var_2)

cat("\nEigenvalues:\n")
print(round(eigenvalues_2, 4))

cat("\nProportion of variance explained:\n")
print(round(prop_var_2 * 100, 2))

cat("\nCumulative proportion of variance:\n")
print(round(cum_prop_var_2 * 100, 2))

cat("\nLoadings (first 2 PCs):\n")
print(round(pca_2$rotation[, 1:2], 4))

# Case 2: PCA with 3 principal components
cat("\n\n=== CASE 2: PCA with 3 Principal Components ===\n\n")

pca_3 <- prcomp(data_df, center = TRUE, scale. = TRUE)

# Summary
cat("Summary of PCA (3 components):\n")
print(summary(pca_3))

# Eigenvalues and variance explained
eigenvalues_3 <- pca_3$sdev^2
prop_var_3 <- eigenvalues_3 / sum(eigenvalues_3)
cum_prop_var_3 <- cumsum(prop_var_3)

cat("\nEigenvalues:\n")
print(round(eigenvalues_3, 4))

cat("\nProportion of variance explained:\n")
print(round(prop_var_3 * 100, 2))

cat("\nCumulative proportion of variance:\n")
print(round(cum_prop_var_3 * 100, 2))

cat("\nLoadings (first 3 PCs):\n")
print(round(pca_3$rotation[, 1:3], 4))

# --- Visualization ---

# Save plots to PDF
pdf("Task2/Task2_Results.pdf", width = 12, height = 8)

# 1. Scree plot
par(mfrow = c(2, 2))

# Scree plot for 2 PCs
plot(1:6, eigenvalues_2, type = "b", 
     main = "Scree Plot (2 PCs Analysis)",
     xlab = "Principal Component", ylab = "Eigenvalue",
     pch = 19, col = "blue")
abline(h = 1, lty = 2, col = "red")
legend("topright", "Kaiser criterion (lambda=1)", lty = 2, col = "red")

# Variance explained plot
barplot(prop_var_2 * 100, names.arg = paste0("PC", 1:6),
        main = "Proportion of Variance Explained (2 PCs)",
        xlab = "Principal Component", ylab = "Variance (%)",
        col = "steelblue", ylim = c(0, max(prop_var_2 * 100) * 1.2))

# Scree plot for 3 PCs
plot(1:6, eigenvalues_3, type = "b", 
     main = "Scree Plot (3 PCs Analysis)",
     xlab = "Principal Component", ylab = "Eigenvalue",
     pch = 19, col = "darkgreen")
abline(h = 1, lty = 2, col = "red")
legend("topright", "Kaiser criterion (lambda=1)", lty = 2, col = "red")

# Cumulative variance plot
barplot(cum_prop_var_3 * 100, names.arg = paste0("PC", 1:6),
        main = "Cumulative Variance Explained (3 PCs)",
        xlab = "Principal Component", ylab = "Cumulative Variance (%)",
        col = "darkgreen", ylim = c(0, 100))

dev.off()

# 2. Biplot
pdf("Task2/Task2_Biplot.pdf", width = 10, height = 8)

par(mfrow = c(1, 2))

# Biplot for 2 PCs
biplot(pca_2, choices = c(1, 2), 
       main = "Biplot: PC1 vs PC2",
       cex = 0.7, scale = 0)

# Biplot for 3 PCs (PC1 vs PC2)
biplot(pca_3, choices = c(1, 2), 
       main = "Biplot: PC1 vs PC2 (3 PCs Analysis)",
       cex = 0.7, scale = 0)

dev.off()

# 3. Correlation plot
pdf("Task2/Task2_Correlation.pdf", width = 8, height = 8)
corrplot(cor(data_df), method = "circle", type = "upper",
         order = "hclust", tl.cex = 0.8, tl.col = "black",
         title = "Correlation Matrix of Original Variables")
dev.off()

# 4. Loadings plot
pdf("Task2/Task2_Loadings.pdf", width = 12, height = 6)

par(mfrow = c(1, 2))

# Loadings for PC1 and PC2
loadings_2 <- pca_2$rotation[, 1:2]
barplot(t(loadings_2), beside = TRUE, 
        main = "Loadings: PC1 and PC2",
        xlab = "Variable", ylab = "Loading",
        col = c("blue", "red"),
        legend = c("PC1", "PC2"),
        names.arg = substr(rownames(loadings_2), 1, 8),
        las = 2, cex.names = 0.7)

# Loadings for PC1, PC2, PC3
loadings_3 <- pca_3$rotation[, 1:3]
barplot(t(loadings_3), beside = TRUE, 
        main = "Loadings: PC1, PC2, and PC3",
        xlab = "Variable", ylab = "Loading",
        col = c("blue", "red", "green"),
        legend = c("PC1", "PC2", "PC3"),
        names.arg = substr(rownames(loadings_3), 1, 8),
        las = 2, cex.names = 0.7)

dev.off()

# --- Save results for LaTeX report ---
sink("Task2/Task2_analysis_results.txt")

cat("=== PRINCIPAL COMPONENTS ANALYSIS RESULTS ===\n\n")
cat("Dataset: Multivariate data with 6 variables\n")
cat("Sample size:", n, "\n\n")

cat("=== CASE 1: 2 Principal Components ===\n\n")
cat("Eigenvalues:\n")
print(round(eigenvalues_2, 4))
cat("\nProportion of variance:\n")
print(round(prop_var_2 * 100, 2))
cat("\nCumulative proportion:\n")
print(round(cum_prop_var_2 * 100, 2))
cat("\nLoadings (PC1, PC2):\n")
print(round(pca_2$rotation[, 1:2], 4))

cat("\n\n=== CASE 2: 3 Principal Components ===\n\n")
cat("Eigenvalues:\n")
print(round(eigenvalues_3, 4))
cat("\nProportion of variance:\n")
print(round(prop_var_3 * 100, 2))
cat("\nCumulative proportion:\n")
print(round(cum_prop_var_3 * 100, 2))
cat("\nLoadings (PC1, PC2, PC3):\n")
print(round(pca_3$rotation[, 1:3], 4))

sink()

cat("\n=== Analysis Complete ===\n")
cat("Results saved to:\n")
cat("  - Task2/Task2_Results.pdf (scree plots)\n")
cat("  - Task2/Task2_Biplot.pdf (biplots)\n")
cat("  - Task2/Task2_Correlation.pdf (correlation matrix)\n")
cat("  - Task2/Task2_Loadings.pdf (loadings plots)\n")
cat("  - Task2/Task2_analysis_results.txt (numerical results)\n")
\end{lstlisting}

\end{document}

